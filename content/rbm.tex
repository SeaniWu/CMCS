\section{Rigid body motion}
{\color{myblue} \textbf{Note:} some of the notations in this part is mixed with the ones used by \emph{Amores} in \underline{Una base para el álgebra lineal}, in which column vectors are \emph{not} represented in {bold}.}\\


\subsection{Basic definitions}
A \textbf{rigid body} is a system of particles of mass $m_{\alpha}\ (\alpha=1,...,N)$ in which the \emph{distance} $|\vec{r}_{\alpha}-\vec{r}_{\beta}|$ between any two particles is \emph{constant}. In other words, a rigid body is a mechanical system of $N$ particles subject to the $N(N-1)/2$ time-independent holonomic constraints (that may or may not be mutually independent).\\
In a generic rigid body it's always possible to construct a set of moving axes $\pazocal{S}$ (\textbf{body axes}), with respect to which all it's particles are \emph{fixed}.\\
A generic rigid body has 6 d. of freedom, 3 for the coordinate of the origin of the body axes, and 3 for the matrix $O(t)\in \text{ SO(3)}$ that relates $\pazocal{S}'$ and $\pazocal{S}$.\\
Centre of mass of a r.b.: $\vec{R}=\frac{1}{M}\scaleint{5ex}\rho(\vec{r})\vec{r}\dd[3]{\vec{r}},$ where $M=\scaleint{5ex}\rho(\vec{r})\dd[3]{\vec{r}}$.




\subsection{Angular momentum and $T$ of a rigid body}
Letting the centre of mass be the origin of $\pazocal{S}$:\\
$\vec{r}_{\alpha}'=\vec{R}+\vec{r}_{\alpha},\ \vec{v}_{\alpha}'=\vec{V}+\omegab\times\vec{r}_{\alpha},\ \vec{P}=M\vec{V}$.\\
$\vec{L}=M\vec{R}\times\vec{V}+\vec{L}_{CM}, \text{ where } \vec{L}_{CM}=\sum\limits_{\alpha}m_{\alpha}\vec{r}_{\alpha}\times(\omegab\times\vec{r}_{\alpha})$\\
$T=\frac{1}{2}M\vec{V}^{2}+T_{\text{rot}}$, where $T_{\text{rot}}=\frac{1}{2}\sum\limits_{\alpha}m_{\alpha}(\omegab\times\vec{r}_{\alpha})^{2}$\\
$\then \vec{L}_{\text{CM}}=\sum\limits_{\alpha}m_{\alpha}[r^{2}_{\alpha}\omegab-(\omegab\cdot\vec{r}_{\alpha})\vec{r}_{\alpha}],\ \ \ \ T_{\text{rot}}=\frac{1}{2}\omegab\cdot\vec{L}_{\text{CM}}$




\subsection{Inertia tensor}
\textbf{Note:} the inertia tensor $I$, without any subscript, is calculated w.r.t. the body axes (with $\text{CM}$ as it's origin). But it can also be calculated w.r.t other points (Steiner's theorem).
\subsubsection*{Definition}
The inertial tensor ($I\in\R^{3\times3}$) is a symmetric matrix with elements:\\
$I_{ij}=\sum\limits_{\alpha}m_{\alpha}(\delta_{ij}r^{2}_{\alpha}-x_{\alpha i}x_{\alpha j})\then \mdoubleline{I_{ii}=m_{\alpha}({x_{\alpha j}^{2}+x_{\alpha k}^{2}})}{I_{ij}=-\sum\limits_{\alpha}m_{\alpha}x_{\alpha i}x_{\alpha j}}$\\
If the r.b. is continuous: $\mdoubleline{I_{ii}=\scaleint{5ex}_{\Omega}\rho(\vec{r})(x_{j}^{2}+x_{k}^{2})\dd[3]{\vec{r}},\ i=1,2,3}{I_{ij}=-\scaleint{5ex}_{\Omega}\rho(\vec{r})x_{i}x_{j}\dd[3]{\vec{r}},\ 1\leq i\neq j\leq 3 }$

\subsubsection*{Application}
$T_{\text{rot}}=\frac{1}{2}\omegab\cdot(I\omegab)=\frac{1}{2}\omega^{\transpose} I\omega\ (\Leftarrow$ here $\omega$ is a column (lin. algebraic) vector.\\
$\then T_{\text{rot}}=\frac{1}{2}I_{\vec{n}}\omega^{2}$, where $I_{\vec{n}}$ is the moment of inertia w.r.t. instantaneous axis of rotation $\vec{n}=\frac{\omegab}{\omega}$



\subsubsection*{Steiner's theorem}
The inertia moment computed w.r.t a point $P$ fixed w.r.t body axes is given as:\\
$(I_{P})_{ij}=\sum\limits_{\alpha}m_{\alpha}(\delta_{ij}\tilde{\vec{r}}^{2}_{\alpha}-\tilde{x}_{\alpha i}\tilde{x}_{\alpha j}).$ Taking into account $\tilde{\vec{r}}_{\alpha}=\vec{r}_{\alpha}-\vec{a},\ \vec{a}=\overrightarrow{CP}$ $\then (I_{P})_{ij}=I_{ij}+M(\vec{a}^{2}\delta_{ij}-a_{i}a_{j}).$\\
If $P$ is also fixed w.r.t $\pazocal{S}'$, then, placing $O'=P$, we can reduce the expressions:
\\
$\vec{V}=\omegab\times\vec{R},\ \ \ \ \vec{v}'_{\alpha}=\omegab\times\vec{r}'_{\alpha},\ \ \ \ \vec{L}=I'\omegab,\ (I'=I_{P}),\ \ \ \ T=\frac{1}{2}\omegab\cdot I'\omegab$.\\
$I'$ can be computed from $I$: $I'_{ij}=I_{ij}+M(\vec{R}^{2}\delta_{ij}-X_{i}X_{j}),$ where $X_{h}$ are components of the vector $\vec{R}$.



\subsubsection*{Principal axes of inertia}
Let $A\in\text{SO(3)}$ be a constant change of basis matrix, and $\tilde{\vec{e}}_{i}=\sum\limits_{j}a_{ji}\vec{e}_{j}$.\\
$\then x=A\tilde{x},\ \ \ \ \tilde{I}=A^{-1}IA=A^{\transpose}IA$.\\
Since $I$ is a symmetric matrix, there is always a matrix $A\in \text{SO}(3)$, s.t. $\tilde{I}=A^{\transpose}IA$ is a diagonal matrix, in other words, the moment of inertia w.r.t the alternate body axes $\tilde{\vec{e}}_{i}$ (which are eigenvectors) is expressed as:\\
$\tilde{I}=
\begin{bmatrix}
I_{1} &  & \\
 & I_{2} & \\
 & & I_{3} \\
\end{bmatrix}$, with $I_{i}$ the eigenvalues corresponding to $\tilde{\vec{e}}_{i}$.\\
In such case, $\tilde{\vec{e}}_{i}$ are \textbf{principal axes of inertia}, and $I_{i}$ are \textbf{principal moments of inertia}, which are the roots of the caracteristic polinomial of $I$.\\
If $\vec{e}_{i}$ are p.a.i. then: $\vec{L}_{\text{CM}}=\sum\limits_{i}I_{i}\omega_{i}\vec{e}_{i},\ \ \ \ T_{\text{rot}}=\frac{1}{2}\sum\limits_{i}I_{i}\omega_{i}^{2}$.\\
If the origin of a set of p.a.i. are also fixed in $\pazocal{S}'$, then $\vec{L}, T$ can be calculated with the expressions above, replacing $I$ by $I'$ w.r.t. the point $O'$.



\subsubsection*{Clasification of r.b. in terms of the multiplicity of the eigenvalues $I_{i}$}
\begin{tabular}{ l | c }
  \textbf{Type of symmetry} & \textbf{Definition}\\[0.1em]
  \hline\\[-2pt]
  Asymmetric tops:  & $I_{j}\not= I_{i},\forall i\neq j,$ \\
  Axially symmetric tops: & $I_{i}= I_{j}\neq I_{k},$ \\
  Spherically symmetric tops:  & $I_{i}=I_{j},\ \forall i,j$ \\
\end{tabular}



\subsubsection*{Deduce $I_{i}$ of a rigid body by its invariance under transformations}
\textbf{Note:} both $\Omega$ and $\rho$ have to be invariant under the transfomations.\\
\begin{tabular}{ l | c }
  \textbf{Inv. under} & \textbf{Behaviour of $I_{i}$}\\[0.1em]
  \hline\\[-2pt]
  $x_{i}\mapsto -x_{i}$ & $I_{ij}=0,\ \forall j\neq i$ \\[2pt]
  $x_{i}\mapsto x_{j}$ & $I_{ii}=I_{jj},\ \ \ \ I_{ik}=I_{jk},\ k\neq i,j$ \\
\end{tabular}\\
In particular, for a \emph{solid of revolution} with homogeneous density, taking ${x_{3}}$ as the axis of rotation and arbitrary perpendicular axes ${x_{1}},{x_{2}}$, it satisfy the following invariances: $x_{1}\mapsto -x_{1},\ x_{2}\mapsto -x_{2},\ x_{1}\mapsto x_{2}$, and thus: \\
$I_{11}\equiv I_{1}=I_{2}\equiv I_{22},\ I_{ij}=0\ (i\neq j)$, with: \\
$I_{1}=\pi\rho\scaleint{5ex}_{z_{1}}^{z_{2}} z^{2}f^{2}(z)\dd{z}+\frac{\pi\rho}{4}\scaleint{5ex}
_{z_{1}}^{z_{2}}f^{4}(z)\dd{z},$\hfill$ I_{3}\equiv I_{33}=\frac{\pi\rho}{2}\scaleint{5ex}_{z_{1}}^{z_{2}}f^{4}(z)\dd{z}.$



\subsection{EOM of a rigid body}
\subsubsection*{EOM in an inertial frame}
If $\vec{F}=0$ then $\vec{N}$ computed at any point in $\pazocal{S}'$ is the same.\\
A r.b. is in equilibrium in an $\pazocal{S}\iff \vec{F}=\vec{N}=0\iff \vec{F}=\vec{N}_{\text{CM}}=0$



\subsubsection*{Forces due to a constant field $\mathbf{f}$}
$\lambda_{\alpha}\equiv$``charge'' of a particle $\then \vec{F}_{\alpha}=\lambda_{\alpha}\vec{f}\then \vec{F}=\sum\limits_{\alpha}=\vec{f}\sum\limits_{\alpha}\lambda_{\alpha}=\Lambda\vec{f},$\\
$ \vec{N}=\Big(\sum\limits_{\alpha}\lambda_{\alpha}\vec{r}'_{\alpha}\Big)\times\vec{f}$. \hfill($\downarrow$ Note that $\vec{X}$ is fixed in the body) \\
If $\Lambda\neq0$, defining $\vec{X}=\frac{1}{\Lambda}\sum\limits_{\alpha}\lambda_{\alpha}\vec{r}'_{\alpha}=\frac{1}{\Lambda}\sum\limits_{\alpha}\lambda_{\alpha}(\vec{R}+\vec{r}_{\alpha})=\vec{R}+\frac{1}{\Lambda}\sum\limits_{\alpha}\lambda_{\alpha}\vec{r}_{\alpha}$\\
we have that $\vec{N}=\vec{X}\times\vec{F}\then $ If $\Lambda\neq0$, then the total torque of the external forces coincides with the torque of the total external forces applied at the \emph{centre of charge}. (For the grav. field $\vec{g}$, $\lambda=m$ and C. charge is equal to CM).\\
Torque of the external forces w.r.t. CM is $\vec{N}_{\text{CM}}=(\vec{X}-\vec{R})\times\vec{F}$ (for $\vec{g}$ it is zero). When $\Lambda=0$ we use the former equation for $\vec{N}$ and the torque of the external forces is independent of the reference point, since $\vec{F}=\Lambda\vec{f}=0$.



\subsubsection*{Euler's equations}
Euler's equations is the system $\mdoubleline{I_{1}\dot{\omega}_{1}-(I_{2}-I_{3})\omega_{2}\omega_{3}=N_{1},}{I_{2}\dot{\omega}_{2}-(I_{3}-I_{1})\omega_{1}\omega_{3}=N_{2},\\I_{3}\dot{\omega}_{3}-(I_{1}-I_{2})\omega_{1}\omega_{2}=N_{3}.}$ where $\vec{N}$ and $I$ are computed w.r.t. CM or a point fixed in the body and $\pazocal{S}'$. $\omegab$ and $\vec{N}$ are in a frame of p.a.i. (in general not inertial).\\
If $\vec{N}=0$ and the origin $O'$ of $\pazocal{S}'$ is a point in the body, $\vec{L}_{O'}\equiv\vec{L}$ and $T$ are conserved. Similarly, if $\vec{N}_{\text{CM}}=0$ then $\vec{L}_{\text{CM}}$ and $T_{\text{rot}}$ are conserved.

\smallskip

\subsection{Inertial motion of a symmetric top}
The angular velocity $\omegab$ of a set of axes w.r.t. to another is \emph{additive}.\\
For an axially ($\vec{e}_{3}$) symmetric r.b., by E's eqns, the following conditions holds: \\
$\omega_{3}$, $\Omega=\frac{I_{3}-I_{1}}{I_{1}}\omega_{3}$, $\omega_{0}=\sqrt{\omega_{1}^{2}+\omega_{2}^{2}}$, $\alpha=\arctan{(\omega_{0}/\omega_{3})}$ constants.\\
In frame of body axes, the vector $\omegab$ rotates around $\vec{e}_{3}$ with $\Omega$ (traces out the \textbf{body cone} fixed w.r.t. body axes). The angular velocity $\Omega>0$ for $I_{3}>I_{1}$, and negative otherwise.\\
Relative to body axes, $\vec{L}$ rotates with $\Omega$ around $\vec{e}_{3}$. $\vec{L}$ fixed relative to $\pazocal{S}'$.\\
Relative to $\pazocal{S}'$, $\omegab$ traces out the \textbf{space cone} around $\vec{L}$ (fixed in $\pazocal{S}'$) with $\Omega_{p}$.\\
The angle between $\vec{e}_{3}$ and $\omegab$ is $\alpha$, between $\vec{L}$ and $\vec{e}_{3}$ is $\theta=\arctan{\frac{I_{1}}{I_{3}}\tan\alpha}$.\\
$\Omega_{p}=\frac{L}{I_{1}}=\omega\sqrt{1+\frac{I_{3}^{2}-I_{1}^{2}}{I_{1}^{2}}\cos^{2}\alpha},\ \omegab=\Omega_{p}\frac{\vec{L}}{L}-\Omega\vec{e}_{3}$.\\
\pgfmathsetmacro{\xrot}{70} 
\pgfmathsetmacro{\zrot}{120}
\tdplotsetmaincoords{\xrot}{\zrot}
\pgfmathsetmacro{\rvec}{1.4} 
\pgfmathsetmacro{\rvecp}{0.9} 
\pgfmathsetmacro{\thetavec}{20} 
\pgfmathsetmacro{\thetavecp}{40} 
\pgfmathsetmacro{\phivec}{120}
\pgfmathsetmacro{\alphago}{\thetavecp-\thetavec}
\pgfmathsetmacro{\tanalpha}{tan(\alphago)}
\pgfmathsetmacro{\cosalphainv}{1/cos(\alphago)}
\pgfmathsetmacro{\cosphi}{cos(\phivec)}
\pgfmathsetmacro{\costheta}{cos(\thetavec)}
\pgfmathsetmacro{\sintheta}{sin(\thetavec)}
\pgfmathsetmacro{\sinphi}{sin(\phivec)}
\begin{tikzpicture}[scale=0.7,tdplot_main_coords]
\coordinate (O) at (0,0,0);
\def\l{0.8}
\def\d{1.8}
\tdplotsetcoord{P}{\rvec}{\thetavec}{\phivec} 
\tdplotsetcoord{Q}{\rvec}{\thetavec}{\phivec-180}
\tdplotsetcoord{S}{\rvecp*\cosalphainv}{\thetavec}{\phivec}
\tdplotsetcoord{T}{\rvecp}{\thetavecp}{\phivec}
\tdplotsetcoord{U}{\rvecp*\cosalphainv}{\thetavecp+\alphago}{\phivec}
\tdplotsetcoord{E}{1.9}{\thetavec}{\phivec}

\draw[color=black] (O) -- (P)  node[near end,anchor=north]{}; 
\draw[color=black] (O) -- (Q)  node[near end,anchor=north]{};

%\draw[dashed] (0,0,0) -- (\rvec,0,0) node[anchor=north east]{};
\draw[] (0,0,0) -- (\d,0,0) node[anchor=north east]{$x'$};
%\draw[dashed] (0,0,0) -- (0,\rvec,0) node[anchor= west]{};
\draw[] (0,0,0) -- (0,\d,0) node[anchor= west]{$y'$};
\draw[] (0,0,0) -- (0,0,\d-0.1) node[anchor=west]{$z'$};

%\tdplotdrawarc[]{(O)}{\rvec}{\zrot-180}{\zrot}{anchor=north}{}
%\tdplotdrawarc[dashed]{(O)}{\rvec}{\zrot}{\zrot+180}{anchor=north}{}

\tdplotdrawarc[color=black]{(Pz)}{\rvec*\sintheta}{0}{360}{anchor=north}{}
%\tdplotdrawarc[]{(O)}{0.2}{0}{\phivec}{anchor=north}{$\varphi$}
\tdplotsetthetaplanecoords{\phivec}
%\tdplotdrawarc[tdplot_rotated_coords]{(0,0,0)}{0.2}{0}{\thetavec}{anchor=east}{$\theta$}
%\draw[dashed] (0,0,0) -- (\rvec*\cosphi,\rvec*\sinphi,0);
%\draw[tdplot_rotated_coords] (\rvec,0,0) arc (0:90:\rvec); 
\tdplotsetrotatedcoords{\phivec}{\thetavecp}{0}
\tdplotsetrotatedcoordsorigin{(O)}
%\draw[color=myblue,tdplot_rotated_coords,->] (0,0,0) -- (\l,0,0) node[anchor=north west]{$\vec{e}_{\theta}$};
%\draw[color=myblue,tdplot_rotated_coords,->] (0,0,0) -- (0,\l,0) node[anchor=west]{$\vec{e}_{\varphi}$};
\draw[color=myblue,tdplot_rotated_coords] (0,0,0)
-- (0,0,\d) node[anchor=south west]{$\vec{e}_{3}$};
\draw[color=myblue,tdplot_rotated_coords] (T)
circle (\rvecp*\tanalpha) node[anchor=south]{};
\draw[color=myblue] (S) -- (O);
\draw[color=myblue] (U) -- (O);
\node[black,tdplot_screen_coords] at (-1,0.5,0){Space cone};
\node[myblue,tdplot_screen_coords] at (1,0.1,0){Body cone};
\draw[color=myblue,-latex,thick] (S) -- (E) node[anchor=south]{$\omega$};
\draw[color=myblue,-latex,thick] (Pz) -- (0,0,1.9) node[anchor=south]{$\vec{L}$};

%\draw[tdplot_screen_coords] (O) circle (\rvec);
\end{tikzpicture}
\qquad
\begin{tikzpicture}[scale=0.7,tdplot_main_coords]
\pgfmathsetmacro{\xrot}{70} 
\pgfmathsetmacro{\zrot}{120}
\tdplotsetmaincoords{\xrot}{\zrot}
\pgfmathsetmacro{\rvec}{1.4} 
\pgfmathsetmacro{\rvecp}{0.5} 
\pgfmathsetmacro{\thetavec}{20} 
\pgfmathsetmacro{\thetavecp}{40} 
\pgfmathsetmacro{\phivec}{120}
\pgfmathsetmacro{\alphago}{\thetavecp+\thetavec}
\pgfmathsetmacro{\tanalpha}{tan(\alphago)}
\pgfmathsetmacro{\cosalphainv}{1/cos(\alphago)}
\pgfmathsetmacro{\cosphi}{cos(\phivec)}
\pgfmathsetmacro{\costheta}{cos(\thetavec)}
\pgfmathsetmacro{\sintheta}{sin(\thetavec)}
\pgfmathsetmacro{\sinphi}{sin(\phivec)}
\coordinate (O) at (0,0,0);
\def\l{0.8}
\def\d{1.8}
\tdplotsetcoord{P}{\rvec}{\thetavec}{\phivec} 
\tdplotsetcoord{Q}{\rvec}{\thetavec}{\phivec-180}
\tdplotsetcoord{S}{\rvecp*\cosalphainv}{-\thetavec}{\phivec}
\tdplotsetcoord{T}{\rvecp}{\thetavecp}{\phivec}
\tdplotsetcoord{U}{\rvecp*\cosalphainv}{\thetavecp+\alphago}{\phivec}
\tdplotsetcoord{E}{2}{-\thetavec}{\phivec}

\draw[color=black] (O) -- (P)  node[near end,anchor=north]{}; 
\draw[color=black] (O) -- (Q)  node[near end,anchor=north]{};

%\draw[dashed] (0,0,0) -- (\rvec,0,0) node[anchor=north east]{};
\draw[] (0,0,0) -- (\d,0,0) node[anchor=north east]{$x'$};
%\draw[dashed] (0,0,0) -- (0,\rvec,0) node[anchor= west]{};
\draw[] (0,0,0) -- (0,\d,0) node[anchor= west]{$y'$};
\draw[] (0,0,0) -- (0,0,\d-0.1) node[anchor=west]{$z'$};

%\tdplotdrawarc[]{(O)}{\rvec}{\zrot-180}{\zrot}{anchor=north}{}
%\tdplotdrawarc[dashed]{(O)}{\rvec}{\zrot}{\zrot+180}{anchor=north}{}

\tdplotdrawarc[color=black]{(Pz)}{\rvec*\sintheta}{0}{360}{anchor=north}{}
%\tdplotdrawarc[]{(O)}{0.2}{0}{\phivec}{anchor=north}{$\varphi$}
\tdplotsetthetaplanecoords{\phivec}
%\tdplotdrawarc[tdplot_rotated_coords]{(0,0,0)}{0.2}{0}{\thetavec}{anchor=east}{$\theta$}
%\draw[dashed] (0,0,0) -- (\rvec*\cosphi,\rvec*\sinphi,0);
%\draw[tdplot_rotated_coords] (\rvec,0,0) arc (0:90:\rvec); 
\tdplotsetrotatedcoords{\phivec}{\thetavecp}{0}
\tdplotsetrotatedcoordsorigin{(O)}
%\draw[color=myblue,tdplot_rotated_coords,->] (0,0,0) -- (\l,0,0) node[anchor=north west]{$\vec{e}_{\theta}$};
%\draw[color=myblue,tdplot_rotated_coords,->] (0,0,0) -- (0,\l,0) node[anchor=west]{$\vec{e}_{\varphi}$};
\draw[color=myblue,tdplot_rotated_coords] (0,0,0)
-- (0,0,\d) node[anchor=south west]{$\vec{e}_{3}$};
\draw[color=myblue,tdplot_rotated_coords] (T)
circle (\rvecp*\tanalpha) node[anchor=south]{};
\draw[color=myblue] (S) -- (O);
\draw[color=myblue] (U) -- (O);
\node[black,tdplot_screen_coords] at (-1,0.5,0){Space cone};
\node[myblue,tdplot_screen_coords] at (1.2,0.4,0){Body cone};
\draw[color=myblue,-latex,thick] (S) -- (E) node[anchor=south]{$\omega$};
\draw[color=myblue,-latex,thick] (Pz) -- (0,0,1.9) node[anchor=south]{$\vec{L}$};

%\draw[tdplot_screen_coords] (O) circle (\rvec);
\end{tikzpicture}



\subsection{Miscellaneous}
In general, for a solid of revolution rolling without sliding down a inclined ($\alpha$) plane around the principal axis $\vec{e}_{3}$, the horizontal acceleration is expressed as: $\ddot{x}=\frac{g\sin\alpha}{1+\frac{I_{3}}{Ma^{2}}}\then$.\\
\subsubsection*{Inertia tensor of some regular bodies}
\begin{tabular}{ l | c | c }
	\textbf{Body} & \textbf{Definition} & \textbf{Inertia tensor}\\[0.1em]
	\hline\\[-2pt]

	\smol{Solid Sphere} & \smol{$x^{2}+y^{2}+z^{2}\leq r^{2}$} & \smol{$\frac{2}{5}mr^{2}
	\begin{bmatrix}
	1 &  & \\
	 & 1 & \\
	 & & 1 \\
	\end{bmatrix}$}\\[7pt]
	
	\smol{Hollow Sphere} & \smol{$x^{2}+y^{2}+z^{2}= r^{2}$} & \smol{$\frac{2}{3}mr^{2}
	\begin{bmatrix}
	1 &  & \\
	 & 1 & \\
	 & & 1 \\
	\end{bmatrix}$}\\[7pt]
	
	\smol{Cuboid} & \smol{$\mdoublelineone{-\frac{d}{2}\leq x\leq \frac{d}{2}\\[1pt]
	-\frac{w}{2}\leq y\leq \frac{w}{2}\\[1pt]
	-\frac{h}{2}\leq y\leq \frac{h}{2}}$} & 
	\smol{$\frac{m}{12}\begin{bmatrix}
h^{2}+d^{2} &  & \\
 & w^{2}+d^{2} & \\
 & & h^{2}+w^{2} \\
\end{bmatrix}$}\\[7pt]
	
	\smol{Cylinder} & \smol{$\mdoublelineone{x^{2}+y^{2}\leq r^{2}\\[1pt]
	-\frac{h}{2}\leq z\leq \frac{h}{2}}$} & 
	\smol{$\frac{m}{12}\begin{bmatrix}
3r^{2}+h^{2} &  & \\
 & 3r^{2}+h^{2} & \\
 & & 6r^{2} \\
\end{bmatrix}$}\\[7pt]

	\smol{Rod about end} & \smol{$\mdoublelineone{x^{2}+y^{2}\leq r^{2}\\[1pt]
	0\leq z\leq h\\[1pt]
	h>> r\\[1pt]
	\omegab\parallel \vec{x}}$} &
	\smol{$\frac{1}{3}ml^{2}\begin{bmatrix}
1 &  & \\
 & 0 & \\
 & & 1 \\
\end{bmatrix}$}\\[7pt]

	\smol{Rod about center} & \smol{$\mdoublelineone{x^{2}+y^{2}\leq r^{2}\\[1pt]
	-\frac{h}{2}\leq z\leq \frac{h}{2}\\[1pt]
	h>> r\\[1pt]
	\omegab\parallel \vec{x}}$} &
	\smol{$\frac{1}{12}ml^{2}\begin{bmatrix}
1 &  & \\
 & 0 & \\
 & & 1 \\
\end{bmatrix}$}
\end{tabular}\\








































