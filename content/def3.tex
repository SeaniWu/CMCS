\newcommand{\partials}[2]{\pdv{#1}{#2}}

\makeatletter 
\renewcommand{\section}{\@startsection{section}{1}{0mm}%
                                {.2ex}%
                                {.2ex}%xion
                                {\color{myblue}\sffamily\small\bfseries}}
\renewcommand{\subsection}{\@startsection{subsection}{1}{0mm}%
                                {.2ex}%
                                {.2ex}%x
                                {\sffamily\fontsize{6}{5}\selectfont\bfseries}}
\renewcommand{\subsubsection}{\@startsection{subsubsection}{1}{0mm}%
                                {.2ex}%
                                {.2ex}%x
                                {\color{myblue}\sffamily\tiny\bfseries}}
\newcommand{\subcolorsection}[1]{%
    \colorbox{green!10}{\parbox[t][0em]{\dimexpr\columnwidth-2\fboxsep}{\thesubsection\ #1}}}
\makeatother
\setlength{\parindent}{0pt}

\newcommand{\dotv}[1]{\dot{\vec{#1}}}
\newcommand{\then}{\Rightarrow}
\newcommand{\R}{\mathbb{R}}
\newcommand{\Q}{\mathbb{Q}}
\newcommand{\C}{\mathbb{C}}
\newcommand{\ext}{^{\text{(e)}}}
\renewcommand{\i}{_{i}}
\renewcommand{\ij}{_{ij}}
\renewcommand{\j}{_{j}}
\renewcommand{\vec}[1]{\mathbf{#1}}


\newcommand{\ELt}[2]{\frac{\text{d}}{\text{d}t}\frac{\partial #1}{\partial \dot{#2}}-\frac{\partial #1}{\partial #2}}

\newcommand{\tqp}[0]{(t,\vec{q},\vec{p})}

\newcommand{\bfemph}[1]{\textbf{\emph{#1}}}

\newcommand{\mdoubleline}[2]{\left\{
\begin{aligned}
#1 \\
#2
\end{aligned}
\right.}

\newcommand{\diff}[1]{\qty(\dv{#1}{t})_{\text{f}}}
\newcommand{\difm}[1]{\qty(\dv{#1}{t})_{\text{m}}}

\makeatletter
\newcommand*{\transpose}{%
  {\mathpalette\@transpose{}}%
}
\newcommand*{\@transpose}[2]{%
  % #1: math style
  % #2: unused
  \raisebox{\depth}{$\m@th#1\intercal$}%
}
\makeatother

\newcommand{\vect}[1]{\boldsymbol{\mathbf{#1}}}
\newcommand{\omegab}[0]{\color{myblue}{\vect{\upomega}}}

\newcommand{\smol}[1]{{\fontsize{4}{5}\selectfont #1 }}
\newcommand{\ssmol}[1]{{\fontsize{3}{4}\selectfont #1 }}


\newcommand{\mdoublelineone}[1]{\left\{
                \begin{array}{ll}
                  #1\\
                \end{array}
              \right.
              }

\renewcommand{\hbar}{\mathchar'26\mkern-9mu h}
































